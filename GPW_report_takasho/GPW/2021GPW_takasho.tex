\documentclass[twocolumn]{ltjsarticle}
\begin{document}
\title{\bf\vspace{-3cm}
{トレーディングカードゲームにおける
\\初期手札枚数差による勝率変化調査} 
}
\author{\vspace{-1cm}髙橋昇太\footnotemark[1] 
阿原一志\footnotemark[1]}
\date{}
\twocolumn[
\maketitle
\small{
\abstractname{:トレーディングカードゲーム(TCG)には,手札枚数や個々のカードの攻撃力など,様々
なパラメータが存在する.これらの値は一般に勝敗に大きく関わるとされているが,科学的実
証は報告されていない.そこで本研究では,パラメータ変更による勝率の変化について調査す
る手法の提案を行う.本論文では特に,初期手札の枚数差を意図的に生じさせ,どのように勝
率が変化するかを調査した.その結果,初期手札枚数が多いプレイヤーは有意に勝率が高いこ
とが明らかになった.}
}
\begin{center}
  \vspace{0.3cm}
{\large \bf{A Study on the Effect of the Number of Cards in the Starting Hand
  \\on the Winning Percentage in Trading Card Games
  }
}
\author{\vspace{0.1cm}\\Shota Takahashi\footnotemark[1]  Kazushi Ahara\footnotemark[1]}
\end{center}
]

\vspace{0.5cm}
\section{はじめに}
トレーディングカードゲーム(以下,TCG)とは
「Magic: the Gathering」などを例とする 2 人用不完全
情報ゲームで,近年では「Hearthstone」や「Shadowverse」
のようにオンライン上で遊べるものもあり,その裾
野が広がっている.TCG は囲碁や将棋のようにター
ン制で進むが,使用するカードを各プレイヤーが選
べ(使用するカードの束をデッキと呼ぶ),デッキから
ランダムに引くカードの種類,いわゆる「引き」によ
って,戦局や戦略が左右されるという点が TCG の大
きな特徴である.
TCG には様々なパラメータが存在する.手札の枚
数やターン数,出したカードの攻撃力値やプレイヤ
ーの体力,デッキの枚数などがそれにあたる.これら
のパラメータの値は対戦結果に大きな影響を及ぼす
とされている.実際に TCG 運営会社は,試合中のパ
ラメータを調整するようなルールやカード能力を設
定している.例えば「Hearthstone」では,先手後手の
有利差を少なくするためにコインというパラメータ
調整カードを後手プレイヤーに初期配布するという
ルールになっている.同様の理由で「Shadowverse」
では,先手プレイヤーは手札を 4 枚,後手プレイヤー
は手札を 5 枚から試合開始となる.またこれらのゲ
ームでは,特定のカードやデッキの使用率,勝率が著
しく高かった場合,ナーフと呼ばれるカード能力の下方修正が行われるケースもある.このように,パラ
メータの値は試合結果に大きな影響を及ぼすことが
わかっている.一方これらの事象(調整ルールやナー
フ)は経験から来る考えであり,定量的な検証は公開
されていない.
そこで本研究では,パラメータの変更による勝率
変化を実験により測定し,実際にどれくらいハンデ
ィキャップが生じるかを統計的に調査する.この手
法を用いることにより,カード能力調整をはじめと
した運営開発に貢献できるほか,TCG 初心者へ向け
た良質なハンデ調整ができるようになると筆者は考
える.本論文では,種種のパラメータの中から特に初
期手札の枚数差による勝率変化について詳細に記す.


\footnotetext[1]{総合数理学部先端メディアサイエンス学科}
\end{document}